\documentclass[a4paper, 12pt]{article}
\usepackage{comment} 
\usepackage{fullpage}
\usepackage[hidelinks]{hyperref}
\usepackage{amsmath}
\usepackage{environ}
\usepackage{tabto,enumitem}

\begin{document}
\noindent
\large\textbf{SOEN 6011 SEP} \hfill \textbf{Mahavir Patel} \\
\normalsize Problem 2 \hfill \textbf{40198619} \\
Function 6 :  $B(x,y)$ Beta Function \hfill Date: 05/07/2022 \\

\section*{Assumptions}
    \begin{enumerate}[noitemsep]
        \item x and y are positive real numbers $x,y \in R^{+}$.
        \item It is simpler to calculate $B(x,y)$ using the factorial for $x,y \in Z^{+}$.
        \item If x and y are real numbers, there is no need to compute the integral function. The gamma values of the numbers may be used to derive the beta value using Stirling's approach. 
    \end{enumerate}
\section*{Requirements}
\begin{enumerate}[noitemsep]
        \item \textbf{First Requirement}
        \begin{itemize}[noitemsep]
            \item \textbf{ID = } FR1
            \item\textbf{Type = } Functional Requirements
            \item\textbf{Version = } 1.0
            \item\textbf{Difficulty = } Easy
            \item\textbf{Description = } The Beta function $B(x,y)$ requires $x$ and $y$ as its two variable inputs in order to operate.
            \item\textbf{Rationalle = } $x$ and $y$ 
        \end{itemize}
        \item \textbf{Second Requirement}
        \begin{itemize}
            \item \textbf{ID = } FR2
            \item\textbf{Type = } Functional Requirements
            \item\textbf{Version = } 1.0
            \item\textbf{Difficulty = } Easy
            \item\textbf{Description = } The Beta function $B(x,y)$ requires two real positive numbers as it's defined in the $R^{+}$ domain.
            \item\textbf{Rationalle = } $x \geq 0$ and $y \geq 0$ 
        \end{itemize}
        \item \textbf{Third Requirement}
        \begin{itemize}
            \item \textbf{ID = } FR3
            \item\textbf{Type = } Functional Requirements
            \item\textbf{Version = } 1.0
            \item\textbf{Difficulty = } Easy
            \item\textbf{Description = } The Beta Value of the function is in real positive numbers i.e $R^{+}$
            \item\textbf{Rationalle = } $B(x,y) \geq 0$ 
        \end{itemize}
        
        \newpage
        \item \textbf{Fourth Requirement}
        \begin{itemize}
            \item \textbf{ID = } FR4
            \item\textbf{Type = } Functional Requirements
            \item\textbf{Version = } 1.0
            \item\textbf{Difficulty = } Easy
            \item\textbf{Description = } If the given inputs are positive integers then beta Values can be easily computed by using the Beta - Gamma Function relation.
            \item\textbf{Rationalle = } \{ \forall x,y \in $Z^{+}$ \mid $B(x,y)$=$\frac{{\Gamma x} {\Gamma y}}{\Gamma (x+y)}$ \}
        \end{itemize}
        \item \textbf{Fifth Requirement}
        \begin{itemize}
            \item \textbf{ID = } FR5
            \item\textbf{Type = } Functional Requirements
            \item\textbf{Version = } 1.0
            \item\textbf{Difficulty = } Moderate
            \item\textbf{Description = } To calculate Beta function for large integer values, Gamma Function should be used in order to prevent stack overflow by using tail recursive function.
            \item\textbf{Rationalle = } \{ \forall x,y \in $R^+$ \mid $B(x,y)$=$\frac{{\Gamma x} {\Gamma y}}{\Gamma (x+y)}$ where $\Gamma n = (n-1)! $ \}
        \end{itemize}
        
        \item \textbf{Sixth Requirement}
        \begin{itemize}
            \item \textbf{ID = } FR6
            \item\textbf{Type = } Functional Requirements
            \item\textbf{Version = } 1.0
            \item\textbf{Difficulty = } Difficult
            \item\textbf{Description = } For the decimal number gamma value can be calculated using the stirlings's approximation which helps in determining the Beta value without using the integral functions.
            \item\textbf{Rationalle = } \{ \forall x,y \in $R^{+}$ \mid $B(x,y)$=$\frac{{\Gamma x} {\Gamma y}}{\Gamma (x+y)}$ where $\Gamma n = \sqrt{2 \cdot \pi \cdot n}\cdot (\frac{n}{e})^{n}$ \} 
        \end{itemize}
        
        \item \textbf{Seventh Requirement}
        \begin{itemize}
            \item \textbf{ID = } FR7
            \item\textbf{Type = } Functional Requirements
            \item\textbf{Version = } 1.0
            \item\textbf{Difficulty = } Moderate
            \item\textbf{Description = } There is no definition of beta values for negative or zero values. There shouldn't be any inputs besides the numeric values; x and y can be similar or different, but there shouldn't be any inputs other than the numeric values.
            \item\textbf{Rationalle = } \{ \forall x,y \in $R^{+}$ \mid\; $x$ > $0$\; and \;$y$ > $0$, $ x = y $ or $ x \neq y $ \}
        \end{itemize}
        \newpage
    \end{enumerate}


\end{document}
